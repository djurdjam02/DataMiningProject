\documentclass[a4paper,12pt]{article}
\usepackage[serbian]{babel}
\usepackage[T1]{fontenc}
\usepackage{lmodern}

\begin{document}

\begin{titlepage}
    \centering
    \vspace*{0.2cm}

    { Univerzitet u Beogradu \\ Matematički fakultet\par}

    \vfill

    {\Large \textbf{Seminarski rad}\par}

    \vspace{1cm}

    {\Large \textbf{Analiza, klasifikacija i klasterovanje proteinskih blokova}\par}

    \vfill

    
	
	
	\begin{tabbing}
	\hspace{10cm} \= \hspace{10cm} \= \kill
	\textbf{Mentor:} \>  \textbf{Studenti:} \\
	Prof. dr Nenad Mitić \> Anja Milutinović 235/2021 \\
	Katedra za računarstvo i informatiku \> Đurđa Milošević 84/2021 \\
	\> Smer: Informatika
	\end{tabbing}

    \vfill

    \textbf{Datum:} 2024/25

\end{titlepage}
\newpage
\tableofcontents
\newpage
\section{Uvod}
U ovom seminarskom radu proučavani su proteinski blokovi dobijeni prevođenjem humanog proteoma formiranog AlphaFold2 programom. Preciznije vršena je analiza neočekivanih prelaza između proteinskih blokova, amino-kiselina i sekundarnih struktura koje se nalaze u tim prelazima, kao i njihove opšte zastupljenosti u podacima. Klasifikacija je urađena nad dva skupa podataka koja sadrže informacije o strukturi proteina tako što je ciljni atribut u prvom skupu bio sam protein dok su se u drugom skupu predviđale amino-kiseline, sekundarne strukture i proteinski blokovi. \\
Seminarski rad je rađen u okviru kursa Istraživanje podataka 2 na Matematičkom fakultetu Univerzitetu u Beogradu.
\newpage
\section{O proteinskim blokovima}
\newpage
\section{Analiza}
\newpage
\section{Klasifikacija}
\newpage
\section{Klasterovanje}
\newpage
\section{Zaključak}
\newpage
\section{Literatura}
\end{document}
